%
%             DO WHAT THE FUCK YOU WANT TO PUBLIC LICENSE
%                     Version 2, December 2004
% 
%  Copyright (C) 2004 Sam Hocevar <sam@hocevar.net>
% 
%  Everyone is permitted to copy and distribute verbatim or modified
%  copies of this license document, and changing it is allowed as long
%  as the name is changed.
% 
%             DO WHAT THE FUCK YOU WANT TO PUBLIC LICENSE
%    TERMS AND CONDITIONS FOR COPYING, DISTRIBUTION AND MODIFICATION
% 
%   0. You just DO WHAT THE FUCK YOU WANT TO. 
%
%   Author : Paul ADENOT
%   Date : 17/10/10 


\documentclass[a4paper, 10pt]{report}

\usepackage[frenchb]{babel}
\usepackage[top=15mm, bottom=15mm, left=30mm, right=30mm]{geometry}
\usepackage[utf8]{inputenc}
\usepackage[T1]{fontenc}
\usepackage{lmodern}
\usepackage{graphicx}
\usepackage{xcolor}
\usepackage{setspace}
\usepackage{sectsty}

\renewcommand*\familydefault{\sfdefault}

\setlength{\parindent}{0pt}

\newcommand{\fontrubrique}[1]{%
    \textsf{\Large{\textbf{#1}}}
}

\newlength{\largeurtitre}
\newlength{\largeurligne}
\newcommand{\rubrique}[1]{%
    \setlength{\largeurtitre}{0pt}
    \setlength{\largeurligne}{0pt}
    \settowidth{\largeurtitre}{\fontrubrique{#1}}
    \addtolength{\largeurligne}{\textwidth}
    \addtolength{\largeurligne}{-\largeurtitre}
    \addtolength{\largeurligne}{-4pt}
    \bigskip
    \fontrubrique{#1} \raisebox{0.3em}{\vrule depth 0pt height 0.6pt width \largeurligne}
}

\newcommand{\titrecadre}[1]{%
    \newlength{\largeurboite}
    \setlength{\largeurboite}{\textwidth}
    \addtolength{\largeurboite}{6cm}
    \hspace{-32mm}
    \begin{minipage}{21cm}
	\fcolorbox{gray}{lightgray}{%
	    \begin{minipage}{\largeurboite}
		\begin{center}
		    \huge{#1}
		\end{center}
	    \end{minipage}
	}
    \end{minipage}
}



\begin{document}
\pagestyle{empty}

\begin{minipage}{0.4\textwidth}
\raggedright
{
        Paul Adenot\\
        105, rue des dames\\
        75017 Paris\\
        France\\
        Tél. Port. : +33~(0)6~27~36~91~36\\
        {\tt <padenot@mozilla.com>}\\
}
\end{minipage}
\hspace{\stretch{1}}
\begin{minipage}{0.5\textwidth}
\raggedleft
{
25 ans\\
Célibataire\\
Nationalité Française\\
}
\end{minipage}

\vfill

\titrecadre{Ingénieur Logiciel Senior}

\vfill

\rubrique{Expériences professionnelles}
\begin{itemize}
  \item Depuis Août 2013: Éditeur de la spécification Web Audio API au World
    Wide Web consortium (W3C).
  \item Depuis Novembre 2012: Ingénieur logiciel chez Mozilla Corporation à
    Paris. Responsable de divers fonctionnalités dans Gecko (Web Audio API,
    sorties et entrées audio).
  \item Été 2012 (6 mois): Stage chez Mozilla Corporation à Mountain View, CA.
    Implementation de divers fonctionalités media, réseau et rendu dans Gecko.
  \item Été 2011 (4 mois): Stage chez Mozilla Europe à Paris. Implémentation de
    divers fonctionalités de {\tt HTMLMediaElement} dans Gecko.
  \item Été 2010 (3 mois): Stage au laboratoire LIRIS, équipe Database (Campus
    LyonTech, 69) : conception et implémentation en \texttt{C++} d'un moteur de
    requêtes continues, basées sur la syntaxe SQL standard.
      \item Été 2009 (3 mois): Stage en entreprise chez ODS-Petrodata à Aberdeen (Écosse). Développement d'un outil de partage de fichiers décentralisé en {\tt C++} sous GNU/Linux.
\end{itemize}
\rubrique{Compétences en informatique}
\begin{itemize}
  \item Langages: {\tt C}, {\tt C++}, HTML, Javascript ({\tt node.js} et
    navigateur), CSS, SQL, \LaTeX{}, Python, Shell.
    \item Programmation système: GNU/Linux, MacOSX, Windows.
    \item Divers bibliothèques spécialisées en audio (VST-SDK, Juce, codecs,
      resamplers, etc.), et API systèmes multimedia.
    \item Moteur de rendu Web (Gecko).
    \item PureData, Max/MSP.
    \item Spécification fonctionelle.
\end{itemize}
\rubrique{Formation}
\begin{itemize}
    \item 2009-2012 : Diplôme d'ingénieur à l'Institut National des Sciences Appliquées (INSA) de Lyon, au département informatique.
    \item 2007-2009 : Diplôme universitaire de technologie (DUT) en informatique à l'institut universitaire de technologie (IUT) de Clermont-Ferrand, spécialité systèmes informatiques.
    \item 2007: Baccalauréat scientifique option Sciences de la Vie et de la
    Terre (SVT), spécialité SVT, mention bien.
\end{itemize}
\rubrique{Langues}
    \begin{itemize}
      \item Anglais: Courant.
      \item Allemand: Notions scolaires.
      \item Espagnol: Notions scolaires.
    \end{itemize}
\rubrique{Autres activités}
\begin{itemize}
    \item Musicien: Guitare classique et électrique (19 ans de pratique), basse, claviers, musique assistée par ordinateur, production~musicale.
    \item Pratique de l'escalade, du cyclisme et du ski alpin.
    \item Vice-président du bureau des étudiants de l'IUT Informatique de Clermont-Ferrand (2008-2009).
    \item Président de de l'association de promotion du logiciel libre de l'INSA de Lyon (Groupe des Pingouins Libres) : organisation de conférences, de campagnes de promotion, et d'install parties.
    \item Responsable de l'équipe Orga'IF (9 personnes) du BdE de l'INSA de Lyon : administration d'un serveur et d'un parc client sous MacOSX, développement d'applications.
\end{itemize}

\vfill

\end{document}
